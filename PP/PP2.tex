\documentclass{article}

\begin{document}

\title{Conflict Management}
\author{Group E}

\maketitle

\section*{Question}
\noindent\fbox{\begin{minipage}{\dimexpr\textwidth-2\fboxsep-2\fboxrule\relax}
You are just promoted and assigned a new CTO {(Chief Information Officer)} by the group of directors of a technology company. There are 2 CTO positions, You manage the \textbf{software} department; another CTO manages the \textbf{hardware} department. 
\end{minipage}}
\noindent\fbox{\begin{minipage}{\dimexpr\textwidth-2\fboxsep-2\fboxrule\relax}
$\textbf{Your Background}$:\\
Your duty is planning the company's software technical vision and leading company's software technological development. Your parents are the biggest shareholders in the group of the directors. Actually, they promoted you. You have limited experiences of management than either old software CTO or hardware CTO. In your previous work, you mainly developed(coded) software project and worked in the flat-management environment.
\end{minipage}}
\noindent\fbox{\begin{minipage}{\dimexpr\textwidth-2\fboxsep-2\fboxrule\relax}
\textbf{Your Conflict}:\\
In yesterday meeting, some software project managers did not agree with your new plan of innovating a software (Finance and Marketing department have shown its predicted profits and budgets are good). They expressed their doubts about your experiences and the feasibility of this plan. They suggested you consult with Hardware CTO. Their attitudes impact the progress of your plan.
\end{minipage}}
\newline \newline
 (a)What the two categories of conflict outcomes would you get from your conflict? Please explain each of them by 1 case.\\
 (b)What does the \textbf{"CUDSA"} stand for?\\
 (c)Identify these project managers by only one of conflict styles.Use the Inclusion Guidelines Planner to solve your conflict(Just write down the main steps that can solve the such style of managers)\\
 (d)Please use the form of "I" statement in CUDSA model to do the 
\textbf{Agree and act} activity for your conflict.\\
 (e)List 1 communication strategies you will apply to communicate with these project managers.\\
 (f)Unfortunately, after your previous actions, you still cannot solve the conflict. Your parents suggest you to involve the third-party negotiation. Pike one of them(Mediator, Arbitrator, Conciliator, Consultant), explain the meaning \\
\section*{Sample Solution}
(a) \\
\textbf{Functional}:Improved managers' performance and attitudes\\
\textbf{DysFunctional}: Cause development of discontent between project managers\\
(b)\\
\textbf{C}onfront the behaviour\\
\textbf{U}nderstand each other's position \\
\textbf{D}efine the problem\\
\textbf{S}earch for a solution\\
\textbf{A}gree and act\\
(c)\\
For example, you picked Competitive {(The Shark)} as the profile of these project managers. This style is: While you're talking, Sharks are thinking of the next argument to defeat you.\\
\textbf{Inclusion guidelines planner}:\\
(1)Define the Situation;\\
(2)Explore Facts, Feelings and Needs;\\
(3)Generate Ideas;(4)Clarify Next Steps Forward.\\
In above steps, you need to give some particular step extension for your project managers' profile you picked. \\
\newline
(d)\\
The sentence style is like: In these (...) circumstances, I agree to ... and you agree to ... If things substantially change then we will review this agreement.\\
\newline
(e)\\
Focus on the project managers' doubt, not the person. Or\\
Choose an appropriate time to go to these projectors' office rooms. Or\\
Give positive as well as negative feedback of their judgements/opinions.\\
\newline
(f)\\
\textbf{1. Mediator}A neutral third party who facilitates a negotiated solution by using reasoning, persuasion, and suggestions for alternatives\\\textbf{2. Arbitrator}A third party to a negotiation who has the authority to dictate an agreement.\textbf{3. Conciliator}A trusted third party who provides an informal communication link between the negotiator and the opponent\textbf{4. Consultant}An impartial third party, skilled in conflict management, who attempts to facilitate creative problem solving through communication and analysis\\
\end{document}
